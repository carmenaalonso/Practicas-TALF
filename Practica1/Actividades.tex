\documentclass[11pt]{article}
%Gummi|065|=)
\title{\textbf{Activities}}
\author{Carmen Alonso Jimenez\\
2º Informatica A}
\date{18/10/2022}
\begin{document}

\maketitle
\section{Find the power set $R^3$ of $R$ = \{(1,1),(1,2),(2,3),(3,4)\}. Check your answer with the script powerrelation.m and write a \LaTeX{} document with the solution step by step}


\begin {center}
Para hallar la potencia tendremos que calcular antes la potencia al cuadrado:

 
$R^2$ = \{(1,1),(1,2),(1,3),(2,4)\}


Desde esta potencia ya sabremos como calcular la potencia al cubo:

$R^3$ = {(1,1),(1,2),(1,3),(1,4)}

\end {center}
\section{Within the folder ``files'', find a TEX file in whose content appears the string code textblash usepackage \{amsthm, amsmath\}. Note: use grep and escape the special characters with textbackslash. Complete the proof and answer the question. }

Consideremos $L=\{w\in \{a,b\}^* : w \textnormal{ no termina en } ab\}$. Un expresión regular que genera L es: \\
\center{$\epsilon + b + (a+b)^*(a+bb)$}

\end{document}
